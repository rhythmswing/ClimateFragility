
\subsection{Temporal Model}
\label{sec:model2}
% In Section~\ref{sec:model1} we analyzed in depth the relationship between the three variables: the environmental factors $\EFF$, the human factors $\HFF$, and the fragility score $\FFF$. 
In order to better model climate change, we further consider time as a variable, and investigate how climate change would have influenced the fragility of states. 

Specifically, we choose China as an example state, and illustrate how and when it would reach a tipping point. The basic idea of our model is as illustrated in the formula below:
\begin{equation}
   E_t = \alpha E_{t-1} + \beta H_{t-1}
   \label{eqn:mod:temporal_idea}
\end{equation}
In which we use the subscript $t$ to denote the value at time $t$.
The first term embodies the autoregressive property of the environmental factor: its present status depends on its previous status. 

\hide{
\begin{equation}
    \EFF = T+S+C+I
   \label{eqn:model:temporal} 
\end{equation}
in which $\EFF$ is the environmental factor, $T$ represents long-term tendency, $S$ represents seasonal fluctuation, $C$ represents periodic fluctuation with an unknown period, and $I$ represents random fluctuation. This is a traditional temporal model; for specifics, readers may refer to~\reminder{reference}.

The long-term tendency $T$ is the crucial ingredient. In the following analysis, we consider the temporal model on a yearly basis, and therefore ignore the seasonal fluctuation $S$. The periodic fluctuation in climate change is majorly manifested in the periodic movement of the solar system~\reminder{reference}, and the period is long enough to ignore in our study which considers only several decades of years. The random fluctuation $I$ is assumed in our analysis to be the white noise. % $\rm{WN}(0,\sigma^2)$.

Therefore, our model is expressed as 
\begin{equation}
    E_t = \alpha +\delta t +\epsilon_t
   \label{eqn:model:temporal:tend} 
\end{equation}
where $t$ indicates time, $E_t$ is the environmental factor at time $t$, $\epsilon_t \sim \rm{N}(0,\sigma^2)$ is a white noise. To estimate $\alpha$ and $\delta$, we utilize tools of traditional temporal analysis, the OLS estimate, to obtain the asymptotic distributions of parameters $\alpha$ and $\delta$.
}

\hide{
为了预测该国家脆弱程度达到临界值的时间,我们首先建立如下具有普适性的时间序列分解模型:
$$Y_t=T+S+C+I$$
其中$Y_t$表示t时刻的脆弱性,T表示长期趋势,即数据中对时间的变化相对稳定的一部分因素,比如在当前的气候条件下下,虽然气温的变化存在波动,但总体存在着上升的趋势。S表示季节变动,传统的时间序列分解方法一般用在长期的宏观经济指标中,因此颗粒度是季度。C是循环变动,循环变动和季节变动比较类似,也有周期因素在,但是循环变动的周期是隐形的。I是随机波动,是不可预测的波动。

具体来说,在气候的分析中,长期趋势是最重要的部分,表示了气候变化的总体趋势,而我们分析的数据是以年为单位的,因此我们不需要考虑数据的季节变动的部分,而气候变化在循环变动这部分主要受太阳系运动的周期变化的影响,其周期比较长,在几十年的气候研究中我们不考虑这一部分对模型的影响。而随机波动是由多种原因综合导致的,通常假定他是白噪声$WN(0,\sigam^2)$。

这样我们就可以建立含确定性时间趋势的过程如下:
$$Y_t=\alpha + \delta t + \epsilon _t$$
其中$\epsilon_t$是白噪声过程,且$\epsilon_t \sim N(0,\sigma ^2)$,则上述模型满足经典回归假设,标准OLS t或F统计量将具有精确小样本t或F分布。但是考虑到气候的复杂性以及我们忽略了气候的循环变动,因此这里的$\epsilon_t$是修正后的误差,不再假定他是高斯分布,则我们使用求$\alpha$和$\delta$的OLS估计的渐进分布的技术,该技术不但对于研究时间趋势有用,而且对于分析各种非平稳过程的统计量也非常有用。
}
