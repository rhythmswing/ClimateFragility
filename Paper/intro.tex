\section{Introduction}	

% introduction to topic.
Climate change has become a common concern for a large portion of the human community. As such, analyzing the effect of climate change and its relation to a state's fragility and people's welfare drew many attentions from researchers.

% What is already done, and the gaps.
Quantifying the fragility of states based on human factors, such as the FSI score has been extensively studied by researchers and instutitions, such as in \cite{FSI_index,EPI_index}. However, these models does not consider the impact of environmental factors. For example, deterioation in natural environment may contribute on regional instability and violence~\cite{schwartz2003abrupt,theisen2013climate,krakowka2012modeling}. As environmental factors are important in determining a state or a region's sustainability, merely considering human factors is clearly insufficient. 

Our work combined previous efforts to incorporate human factors and environmental factors into a novel fragility score that is consistent with traditional results. We also analyzed the effects of environmental factors, both indirect and direct. We also forecasted the future climate change of an example country, and found that moderate economic development balances environmental damage in the long run. Our results are insightful for policy-makers.

First we propose a theoretical framework of our model in Section~\ref{sec:model}. Then, implementation of our framework, experiment designs and results are thoroughly listed and discussed in detail in Section~\ref{sec:exp}. We discuss the strength and weakness of our model, parameter sensitivity, and the relation of our work to interdiscriplinary works in the field of environmental science in Section~\ref{sec:disc}.