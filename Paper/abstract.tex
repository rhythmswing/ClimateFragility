\begin{abstract}
The center of our work is to model the interaction between human and environmental factors and their joint contribution to a state's fragility.
In order to guarantee the interpretability and verifiability of our model, we specifically formulated a concrete theoretical model framework that is flexible and applicable to many stiuations. We then implemented the framework in details.

We proposed a probabilistic model framework in Section~\ref{sec:model:frag} for scoring a country's fragility, combining current well-recognized methods to identify a country's extent of political and environmental fragility. Our fragility score is consistent with traditional methods, and yields more comprehensive information about a country's fragility.

We were then able to dive into the complex dynamics between factors. In Section~\ref{sec:model:indirect}, we used theoretically well supported and verifiable tools, e.g. PSM, to measure the impact of direct environmental factors on a state's fragility. We proposed and verified two hypothesis models for indirect effects: the mediator variable model and the moderator variable model. A case study of Iraq gives us insight in how the dynamics take place.

We further developed a temporal model in Section~\ref{sec:model2} to describe the dynamics of environmental change with the presence of human intervention. Specifically, we use ARIMA, a widely used time series model to forecast a country's GDP growth, and based on this, we modeled climate change based on its autoregressive property and its direct and indirect relation between GDP.

Our model produced fruitful and insightful results. For example, we found the different patterns of impact of climate change for fragile and stable states in Section~\ref{sec:exp:indirect}, and derived the interaction between envrionment and multiple human factors consistent with interdisciplinary knowledge. 
The results are informative for policy making process. 
In Section~\ref{sec:exp:temporal}, we discussed the trade-off rapid development and environmental protection, and discovered that mediocre economic growth can balance environmental deterioration in the long term. 
\end{abstract}