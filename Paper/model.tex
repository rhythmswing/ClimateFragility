\section{Theoretical Analysis}
In this part we propose theoretical framework for our analysis of the impact of climate change on \emph{state fragility}.
\subsection{Assumptions and Model Framework}
We propose two natural assumptions, based on which we derive the basic framework of our model. 
\begin{enumerate}
   \item The \emph{state fragility}, a concept to estimate the sustainability of states, is dependent on and only on \emph{human factors} and \emph{environmental factors.} \label{model:assump:1}
   \item The environmental factors and human factors interact with each other. \label{model:assump:2}
\end{enumerate}
The assumptions are natural. Assumption~\ref{model:assump:1} requires to quantify the fragility which considers both human factors and environmental factors. We propose a novel framework to quantify fragility by incorporating the human and environmental factors into a probabilistic framework.

Assumption~\ref{model:assump:2} requires a more sophisticated analysis of the two factors, including their respective and joint effects on the fragility, and the interaction between them. We are especially interested in the effects of environmental factors, which include \emph{direct} effect, which is the influence on fragility directly imposed by environmental factors; and \emph{indirect} effect, which is the influence on fragility imposed by envirionmental factors indirectly through human factors. The model is visualized in \reminder{A graphical illustration}.

In the following of this section, we are dedicated to enrich our model.

\subsection{Representing the Two Factors}

\subsection{Probabilistic Fragility Measure}
In this part, we propose a novel method to quantify the fragility of a state or region, by considering both human and environmental factors in a probabilistic framework.

\subsection{The Relation between Factors and Fragility}
In this part, we quantify the relationship between human factors and environmental factors. 

\subsection{Measuring Direct and Indirect Effects}
% direct: propensity 
% indirect: 调节效应,中介效应. 