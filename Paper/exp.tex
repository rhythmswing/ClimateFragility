\section{Experiments}
\label{sec:exp}
\subsection{Data Preparation}
\label{sec:exp:prep}
We prepared several representative datasets for the variables we need: $\EFF$, $\HFF$, and $\FFF$.

\subsection{Direct Effect of Environmental Factors}
\label{sec:exp:direct}

\subsection{Indirect Effect of Environmental Factors}
\label{sec:exp:indirect}
To measure the indirect effect of environmental factors, we modeled several regression models in Section~\ref{sec:model:indirect}. Specifically, we choose the EPI index and FSI index described in Section~\ref{sec:exp:prep} to represent the environmental factors $\EFF$ and human factors $\HFF$, respectively. The indexes are scaled and centralized, such that a comparsion of coefficients is meaningful.
\begin{table}[htbp]
    \centering
   \begin{tabular}{|l|ccc|} \hline
      Variable & $\EFF$ & $\HFF$ & $\EFF\times\HFF$ \\ \hline
      Coefficient & 0.17 & 0.91 & 0.06 \\ \hline
      P-value & 0.31 & 0.00 & 0.69679 \\ \hline
      Significant & No & Yes & No  \\ \hline
   \end{tabular} 
   \caption{Moderator variable model of human factors.}
   \label{tab:exp:moderator}
\end{table}

\vpara{Moderator Effect.}
The regression parameters and their p-value is listed in Table~\ref{tab:exp:moderator}. Lower p-value indicates that the original hypothesis that was tested is insignificant. We determine that if the p-value lower a certain threshold, $0.05$, we would reject the hypothesis. Since our hypothesis detailed in Section~\ref{sec:model:indirect} is that the coefficient is zero, p-value lower than $0.05$ would show that the effect of the corresponding variable is significant.

Table~\ref{tab:exp:moderator} shows that the moderator effect is not significant; therefore we would reject the hypothesis that the human factors act as a moderator of the relation between the environmental factors and the fragility.

\begin{table}[htbp]
    \centering
    \begin{tabular}{|l|cccc|} \hline
        Variable & Model & Coef. & p-val. & Res.  \\ \hline
        % \multirow{2}{*}{Human factors} & $\FFF\sim\HFF$ & -0.8349 & $<2e-16$ & Yes \\ \cline{2-5}
        Human factors & $\FFF\sim\HFF + \EFF$ & 1.009 & $<2e-16$ & Yes  \\ \hline        
        \multirow{3}{*}{Environmental factors} & $\HFF\sim\EFF$ & -0.8349 & $<2e-16$ & Yes \\ \cline{2-5}
        & $\FFF\sim\EFF$ & -0.6152 & $<2e-16$ & Yes \\ \cline{2-5}
        & $\FFF\sim\HFF+\EFF$ & 0.2272 & 0.00773 & Yes \\ \hline        
    \end{tabular}
    \caption{Mediator variable effect of human factors. The column Model tells the regression model, in which the left hand of $\sim$ is the dependent variable, and the right hand are the independent variables used in the regression model.}
    \label{tab:exp:mediator:general}
\end{table}
    
\vpara{Mediator Effect.}
Table~\ref{tab:exp:mediator:general} shows the result of the test of the mediator effect of human factors, represented by the FSI index. The results show that the effect of human factors on fragility is significant, and that the effect of environmental factors on fragility is significant, by corresponding coefficients in models $\FFF\sim\HFF+\EFF$ and $\HFF\sim\EFF$. In both models of $\FFF\sim\EFF$ and $\FFF\sim\HFF+\EFF$, the effects of human factors on the fragility are significant; furthermore, the absolute value of the coefficient is reduced by approximately $0.4$ once $\HFF$ is introduced. Therefore, the human factors act as a mediator variable, through which the environmental factors influence fragility. Furthermore, model $\FFF\sim\EFF$ shows that environmental factors also influence fragility directly. As such, a statistical proof of the mediator variable model in Section~\ref{sec:model:indirect} is complete.

\vpara{Conclusion on Indirect Effects.} Above experiments show that:
\begin{itemize}
   \item The environmental factors and human factors both directly influence fragility;
   \item Environmental factors also influence human factors, and human factors act as a mediator variable to pass the influence of the environmental factors to fragility. 
\end{itemize}
Which is illustrated in Figure~\ref{fig:model:indirect:mediator}. 

\subsection{Temporal Model}
\label{sec:exp:temporal}

\subsection{Regional Model}
\label{sec:exp:regional}